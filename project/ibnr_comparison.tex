% Options for packages loaded elsewhere
\PassOptionsToPackage{unicode}{hyperref}
\PassOptionsToPackage{hyphens}{url}
%
\documentclass[
]{article}
\usepackage{amsmath,amssymb}
\usepackage{iftex}
\ifPDFTeX
  \usepackage[T1]{fontenc}
  \usepackage[utf8]{inputenc}
  \usepackage{textcomp} % provide euro and other symbols
\else % if luatex or xetex
  \usepackage{unicode-math} % this also loads fontspec
  \defaultfontfeatures{Scale=MatchLowercase}
  \defaultfontfeatures[\rmfamily]{Ligatures=TeX,Scale=1}
\fi
\usepackage{lmodern}
\ifPDFTeX\else
  % xetex/luatex font selection
\fi
% Use upquote if available, for straight quotes in verbatim environments
\IfFileExists{upquote.sty}{\usepackage{upquote}}{}
\IfFileExists{microtype.sty}{% use microtype if available
  \usepackage[]{microtype}
  \UseMicrotypeSet[protrusion]{basicmath} % disable protrusion for tt fonts
}{}
\makeatletter
\@ifundefined{KOMAClassName}{% if non-KOMA class
  \IfFileExists{parskip.sty}{%
    \usepackage{parskip}
  }{% else
    \setlength{\parindent}{0pt}
    \setlength{\parskip}{6pt plus 2pt minus 1pt}}
}{% if KOMA class
  \KOMAoptions{parskip=half}}
\makeatother
\usepackage{xcolor}
\usepackage[margin=1in]{geometry}
\usepackage{color}
\usepackage{fancyvrb}
\newcommand{\VerbBar}{|}
\newcommand{\VERB}{\Verb[commandchars=\\\{\}]}
\DefineVerbatimEnvironment{Highlighting}{Verbatim}{commandchars=\\\{\}}
% Add ',fontsize=\small' for more characters per line
\usepackage{framed}
\definecolor{shadecolor}{RGB}{248,248,248}
\newenvironment{Shaded}{\begin{snugshade}}{\end{snugshade}}
\newcommand{\AlertTok}[1]{\textcolor[rgb]{0.94,0.16,0.16}{#1}}
\newcommand{\AnnotationTok}[1]{\textcolor[rgb]{0.56,0.35,0.01}{\textbf{\textit{#1}}}}
\newcommand{\AttributeTok}[1]{\textcolor[rgb]{0.13,0.29,0.53}{#1}}
\newcommand{\BaseNTok}[1]{\textcolor[rgb]{0.00,0.00,0.81}{#1}}
\newcommand{\BuiltInTok}[1]{#1}
\newcommand{\CharTok}[1]{\textcolor[rgb]{0.31,0.60,0.02}{#1}}
\newcommand{\CommentTok}[1]{\textcolor[rgb]{0.56,0.35,0.01}{\textit{#1}}}
\newcommand{\CommentVarTok}[1]{\textcolor[rgb]{0.56,0.35,0.01}{\textbf{\textit{#1}}}}
\newcommand{\ConstantTok}[1]{\textcolor[rgb]{0.56,0.35,0.01}{#1}}
\newcommand{\ControlFlowTok}[1]{\textcolor[rgb]{0.13,0.29,0.53}{\textbf{#1}}}
\newcommand{\DataTypeTok}[1]{\textcolor[rgb]{0.13,0.29,0.53}{#1}}
\newcommand{\DecValTok}[1]{\textcolor[rgb]{0.00,0.00,0.81}{#1}}
\newcommand{\DocumentationTok}[1]{\textcolor[rgb]{0.56,0.35,0.01}{\textbf{\textit{#1}}}}
\newcommand{\ErrorTok}[1]{\textcolor[rgb]{0.64,0.00,0.00}{\textbf{#1}}}
\newcommand{\ExtensionTok}[1]{#1}
\newcommand{\FloatTok}[1]{\textcolor[rgb]{0.00,0.00,0.81}{#1}}
\newcommand{\FunctionTok}[1]{\textcolor[rgb]{0.13,0.29,0.53}{\textbf{#1}}}
\newcommand{\ImportTok}[1]{#1}
\newcommand{\InformationTok}[1]{\textcolor[rgb]{0.56,0.35,0.01}{\textbf{\textit{#1}}}}
\newcommand{\KeywordTok}[1]{\textcolor[rgb]{0.13,0.29,0.53}{\textbf{#1}}}
\newcommand{\NormalTok}[1]{#1}
\newcommand{\OperatorTok}[1]{\textcolor[rgb]{0.81,0.36,0.00}{\textbf{#1}}}
\newcommand{\OtherTok}[1]{\textcolor[rgb]{0.56,0.35,0.01}{#1}}
\newcommand{\PreprocessorTok}[1]{\textcolor[rgb]{0.56,0.35,0.01}{\textit{#1}}}
\newcommand{\RegionMarkerTok}[1]{#1}
\newcommand{\SpecialCharTok}[1]{\textcolor[rgb]{0.81,0.36,0.00}{\textbf{#1}}}
\newcommand{\SpecialStringTok}[1]{\textcolor[rgb]{0.31,0.60,0.02}{#1}}
\newcommand{\StringTok}[1]{\textcolor[rgb]{0.31,0.60,0.02}{#1}}
\newcommand{\VariableTok}[1]{\textcolor[rgb]{0.00,0.00,0.00}{#1}}
\newcommand{\VerbatimStringTok}[1]{\textcolor[rgb]{0.31,0.60,0.02}{#1}}
\newcommand{\WarningTok}[1]{\textcolor[rgb]{0.56,0.35,0.01}{\textbf{\textit{#1}}}}
\usepackage{graphicx}
\makeatletter
\def\maxwidth{\ifdim\Gin@nat@width>\linewidth\linewidth\else\Gin@nat@width\fi}
\def\maxheight{\ifdim\Gin@nat@height>\textheight\textheight\else\Gin@nat@height\fi}
\makeatother
% Scale images if necessary, so that they will not overflow the page
% margins by default, and it is still possible to overwrite the defaults
% using explicit options in \includegraphics[width, height, ...]{}
\setkeys{Gin}{width=\maxwidth,height=\maxheight,keepaspectratio}
% Set default figure placement to htbp
\makeatletter
\def\fps@figure{htbp}
\makeatother
\setlength{\emergencystretch}{3em} % prevent overfull lines
\providecommand{\tightlist}{%
  \setlength{\itemsep}{0pt}\setlength{\parskip}{0pt}}
\setcounter{secnumdepth}{-\maxdimen} % remove section numbering
\ifLuaTeX
  \usepackage{selnolig}  % disable illegal ligatures
\fi
\usepackage{bookmark}
\IfFileExists{xurl.sty}{\usepackage{xurl}}{} % add URL line breaks if available
\urlstyle{same}
\hypersetup{
  pdftitle={Comparación métodos IBNR deterministicos y estocásticos},
  pdfauthor={we},
  hidelinks,
  pdfcreator={LaTeX via pandoc}}

\title{Comparación métodos IBNR deterministicos y estocásticos}
\author{we}
\date{}

\begin{document}
\maketitle

\section{Introducción}\label{introducciuxf3n}

\section{Objetivo}\label{objetivo}

\section{Tratamiento de la base}\label{tratamiento-de-la-base}

La base de datos sobre los siniestros de 2015 a 2023 presenta las
siguientes variables:

\begin{itemize}
\tightlist
\item
  \texttt{AÑO}: Indica el año en que ocurrió el siniestro
\item
  \texttt{MONEDA}: Moneda con la cual se emitió la póliza.
\item
  \texttt{FORMA\ DE\ VENTA}: Indica la manera que se vendió la póliza
\item
  \texttt{GIRO}: Tipo de empresa
\item
  \texttt{NUMERO\ DE\ NIVELES}: Número total de niveles del edificio,
  incluyendo sótanos
\item
  \texttt{TIPO\ PRIMER\ RIESGO}: Manera en que está operando el primer
  riesgo.

  \begin{itemize}
  \tightlist
  \item
    Absoluto: Cuando el límite máximo de responsabilidad es inferior al
    valor de los bienes asegurables y se desconoce el valor de los
    bienes
  \item
    Relativo: Cuando el límite máximo de responsabilidad es inferior al
    valor de los bienes asegurables y se conoce el valor de los bienes
  \end{itemize}
\item
  \texttt{ENTIDAD}: la entidad federativa-municipio en donde se
  encuentra el bien o inmueble asegurado
\item
  \texttt{SUBTIPO\ DE\ SEGURO}: Subtipo de seguro que corresponda al
  negocio asegurado

  \begin{itemize}
  \tightlist
  \item
  \end{itemize}
\end{itemize}

\section{Explorando la base de datos de los siniestros por
incendio}\label{explorando-la-base-de-datos-de-los-siniestros-por-incendio}

\section{Construcción del triángulo de pérdidas
acumuladas}\label{construcciuxf3n-del-triuxe1ngulo-de-puxe9rdidas-acumuladas}

La base de datos que la Comisión Nacional de Seguros y Fianzas
proporciona al público en general contiene las siguientes variables

position the caret at any line or the code chunk, then click ``+''.

The code chunk appears:

Type any R code in the chunk, for example:

\begin{Shaded}
\begin{Highlighting}[]
\NormalTok{mycars }\OtherTok{\textless{}{-}} \FunctionTok{within}\NormalTok{(mtcars, \{ cyl }\OtherTok{\textless{}{-}} \FunctionTok{ordered}\NormalTok{(cyl) \})}
\NormalTok{mycars}
\end{Highlighting}
\end{Shaded}

\begin{verbatim}
##                      mpg cyl  disp  hp drat    wt  qsec vs am gear carb
## Mazda RX4           21.0   6 160.0 110 3.90 2.620 16.46  0  1    4    4
## Mazda RX4 Wag       21.0   6 160.0 110 3.90 2.875 17.02  0  1    4    4
## Datsun 710          22.8   4 108.0  93 3.85 2.320 18.61  1  1    4    1
## Hornet 4 Drive      21.4   6 258.0 110 3.08 3.215 19.44  1  0    3    1
## Hornet Sportabout   18.7   8 360.0 175 3.15 3.440 17.02  0  0    3    2
## Valiant             18.1   6 225.0 105 2.76 3.460 20.22  1  0    3    1
## Duster 360          14.3   8 360.0 245 3.21 3.570 15.84  0  0    3    4
## Merc 240D           24.4   4 146.7  62 3.69 3.190 20.00  1  0    4    2
## Merc 230            22.8   4 140.8  95 3.92 3.150 22.90  1  0    4    2
## Merc 280            19.2   6 167.6 123 3.92 3.440 18.30  1  0    4    4
## Merc 280C           17.8   6 167.6 123 3.92 3.440 18.90  1  0    4    4
## Merc 450SE          16.4   8 275.8 180 3.07 4.070 17.40  0  0    3    3
## Merc 450SL          17.3   8 275.8 180 3.07 3.730 17.60  0  0    3    3
## Merc 450SLC         15.2   8 275.8 180 3.07 3.780 18.00  0  0    3    3
## Cadillac Fleetwood  10.4   8 472.0 205 2.93 5.250 17.98  0  0    3    4
## Lincoln Continental 10.4   8 460.0 215 3.00 5.424 17.82  0  0    3    4
## Chrysler Imperial   14.7   8 440.0 230 3.23 5.345 17.42  0  0    3    4
## Fiat 128            32.4   4  78.7  66 4.08 2.200 19.47  1  1    4    1
## Honda Civic         30.4   4  75.7  52 4.93 1.615 18.52  1  1    4    2
## Toyota Corolla      33.9   4  71.1  65 4.22 1.835 19.90  1  1    4    1
## Toyota Corona       21.5   4 120.1  97 3.70 2.465 20.01  1  0    3    1
## Dodge Challenger    15.5   8 318.0 150 2.76 3.520 16.87  0  0    3    2
## AMC Javelin         15.2   8 304.0 150 3.15 3.435 17.30  0  0    3    2
## Camaro Z28          13.3   8 350.0 245 3.73 3.840 15.41  0  0    3    4
## Pontiac Firebird    19.2   8 400.0 175 3.08 3.845 17.05  0  0    3    2
## Fiat X1-9           27.3   4  79.0  66 4.08 1.935 18.90  1  1    4    1
## Porsche 914-2       26.0   4 120.3  91 4.43 2.140 16.70  0  1    5    2
## Lotus Europa        30.4   4  95.1 113 3.77 1.513 16.90  1  1    5    2
## Ford Pantera L      15.8   8 351.0 264 4.22 3.170 14.50  0  1    5    4
## Ferrari Dino        19.7   6 145.0 175 3.62 2.770 15.50  0  1    5    6
## Maserati Bora       15.0   8 301.0 335 3.54 3.570 14.60  0  1    5    8
## Volvo 142E          21.4   4 121.0 109 4.11 2.780 18.60  1  1    4    2
\end{verbatim}

Now, click the \textbf{Run} button on the chunk toolbar to
\href{https://www.jetbrains.com/help/pycharm/r-markdown.html\#run-r-code}{execute}
the chunk code. The result should be placed under the chunk. Click the
\textbf{Knit and Open Document} to build and preview an output.

\end{document}
